\documentclass[12pt]{article}

\usepackage[utf8]{inputenc}
\usepackage{comment}
\usepackage{hyperref}
\usepackage{listings}
\usepackage{graphicx}
\usepackage{titlesec}
\usepackage{hyperref}
\usepackage{blindtext}
\usepackage{enumitem}
\usepackage{fancyhdr}
\usepackage{afterpage}

\pagestyle{fancy}
\fancyhf{}
\lhead{\rightmark}
\rhead{Page \thepage}
%\cfoot{\thepage}

\newcommand\blankpage{%
    \null
    \thispagestyle{empty}%
    \addtocounter{page}{-1}%
    \newpage}

\titleclass{\subsubsubsection}{straight}[\subsection]

\newcounter{subsubsubsection}[subsubsection]
\renewcommand\thesubsubsubsection{\thesubsubsection.\arabic{subsubsubsection}}
\renewcommand\theparagraph{\thesubsubsubsection.\arabic{paragraph}} % optional; useful if paragraphs are to be numbered

\titleformat{\subsubsubsection}
  {\normalfont\normalsize\bfseries}{\thesubsubsubsection}{1em}{}
\titlespacing*{\subsubsubsection}
{0pt}{3.25ex plus 1ex minus .2ex}{1.5ex plus .2ex}

\makeatletter
\renewcommand\paragraph{\@startsection{paragraph}{5}{\z@}%
  {3.25ex \@plus1ex \@minus.2ex}%
  {-1em}%
  {\normalfont\normalsize\bfseries}}
\renewcommand\subparagraph{\@startsection{subparagraph}{6}{\parindent}%
  {3.25ex \@plus1ex \@minus .2ex}%
  {-1em}%
  {\normalfont\normalsize\bfseries}}
\def\toclevel@subsubsubsection{4}
\def\toclevel@paragraph{5}
\def\toclevel@paragraph{6}
\def\l@subsubsubsection{\@dottedtocline{4}{7em}{4em}}
\def\l@paragraph{\@dottedtocline{5}{10em}{5em}}
\def\l@subparagraph{\@dottedtocline{6}{14em}{6em}}
\makeatother

\setcounter{secnumdepth}{4}
\setcounter{tocdepth}{4}

 \hypersetup{
    colorlinks=true,
    linkcolor=black,
    filecolor=magenta,      
    urlcolor=cyan,
}

\urlstyle{same}

\title{\bf{Intelligence Transportation System}}
\author{
  Dr.Emad Nabil\thanks{Thank you for help us to make this project done}\\
  \texttt{professor in cairo university, egypt}
  \and
  Omer elqubati\\
  \texttt{omer.alqubati@gmail.com}
  \and
  Mahmoud Mosaad\\
  \texttt{mahmoudmosaad50@gmail.com}
  \and
  Zyad Aamer\\
  \texttt{ziadaamer9@gmail.com}
  \and
  Hameed\\
  \texttt{hameedxx@gmail.com}
  \and
  Aseel\\
  \texttt{asseel7723@gmail.com}
}
\date{\today }

\begin{document}

\begin{titlepage}
\clearpage\maketitle
\thispagestyle{empty}
\end{titlepage}

\blankpage
%\href{https://github.com/mahmoud-mosaad/transportation-system}{github-fork}
\tableofcontents
\newpage

\section{\sc Project Idea}
Our project idea is to solve the Travelling Salesman problem, this problem appears in Companies that have transportation system and want to minimize the cost of delivering the orders, and this is done by providing the correct order in which distributers should deliver orders.

\section{\sc Problem Significance}

\subsection{ Problem Definition }
The traveling salesman problem is a common NP hard problem and consists of a salesman and a set of cities. The salesman has to visit each one of the cities starting from a certain one (e.g. the hometown) and returning to the same city. The challenge of the problem is that the traveling salesman wants to minimize the total length of the trip.

\subsection{ The Problem variants }

\subsubsection{ The TSP with precedencies or time windows }
A Travelling Salesman Problem with Allocation, Time Window and Precedence Constraints (TSP-ATWPC) is considered. The TSP-ATWPC occurs as a sub problem of optimally sequencing a given set of port visits in a real bulk ship scheduling problem, which is a combined multi-ship pickup and delivery problem with time windows and multi-allocation problem. Each ship in the fleet is equipped with a flexible cargo hold that can be partitioned into several smaller holds in a given number of ways, thus allowing multiple products to be carried simultaneously by the same ship.

\subsubsection{ Symmetric (STSP) }
The TSP is symmetric if, for every pair of cities i and j, the distance from i to j is the same as the one from j to i.

\subsubsection{ Asymmetric (ATSP) }
The TSP is asymmetric if, the distance for going from a point to another may be different of the returning distance.

\subsubsection{ The price collecting TSP }
In the TSP, the salesman has to visit a set of cities while minimizing the length of the overall tour. In the PCTSP, each city has a given weight and penalty, and the goal is to collect a given quota of the weights of the cities while minimizing the length of the tour plus the penalties of the cities not in the tour.

\subsubsection{ The online TSP }
The number of requests n is not known to the online server. Requests are revealed to the online server at their release dates.

\subsubsection{ Bus, truck, vehicle routing }
Asks "What is the optimal set of routes for a fleet of vehicles to traverse in order to deliver to a given set of customers?". It generalizes the well-known travelling salesman problem (TSP).

\subsubsection{ Edge/arc and node routing with capacities }
The CARP aims to find a set of vehicle trips with minimum cost, such that each trip starts and ends at a depot node v0 2 V, each required edge.is serviced by a single trip, and the total demand for any vehicle does not exceed a capacity Q.

\subsubsection{  The symmetric and asymmetric m-TSP }
(m-TSP) is a generalization of the TSP in which more than one salesman is allowed. Given a set of cities, one depot, the objective of the m-TSP is to determine a set of routes for mm salesmen so as to minimize the total cost of the routes. The m-TSP is a relaxation of the vehicle routing problem (VRP); if the vehicle capacity in the VRP is a sufficiently large value so as not to restrict the vehicle capacity, then the problem is the same as the m-TSP.


\subsection{ Motivation }
Assume that you are the driver of a delivery vehicle with a certain set of stops that need to be made each day. How would you determine the order in which to make the stops? If you are interested solely in distance, you could create a graph of the transportation network and weight each edge as the distance of the roadway it represents, allowing a solution to the Traveling Salesman Problem (TSP) to determine the shortest route.

\subsection{ Problem Solution }
Travel Salesman Problem has more than one approach to solve it.\\
e.g:  Exact Approach, Greedy Approach, Heuristic Approach and Genetic Algorithm Approach.

\subsubsection{ Exact Approach }

\subsubsubsection{  Brute Force }
\begin{center}
	\includegraphics[width=6cm,height=14cm]{./assets/flowchart/brute-force.png}\\
\end{center}

\subsubsubsection{  Dynamic Programming (Held Karp) }
\begin{center}
	\includegraphics[width=10cm,height=18cm]{./assets/flowchart/held-karp.png}\\
\end{center}

\subsubsection{ Greedy Approach }

\subsubsubsection{ Nearest Neighbor Algorithm }
\begin{center}
	\includegraphics[width=10cm,height=17cm]{./assets/flowchart/greedy.png}\\
\end{center}

\subsubsection{ Heuristic Approach }
\begin{itemize}
\item \bf{Approximation}: 
Solving the TSP optimally takes to long, instead uses approximation algorithms, or heuristics and can get good solutions but may not optimal.
\item \bf{Tour Construction}:
Tour construction algorithms have one thing in commmon,
they stop when a solution is found and never
tries to improve it.
\item \bf{Tour Improvement}:
Once a tour has been generated by some tour construction
heuristic, we might wish to improve that solution.
There are several ways to do this, but the most
common ones are the 2-opt or k-opt local searches.
Their performances are somewhat linked to the construction
heuristic used.
\end{itemize}

\subsubsubsection{ Cheapest Link (Construction)}
\begin{center}
	\includegraphics[width=11cm,height=14cm]{./assets/flowchart/cheapest.png}\\
\end{center}

\newpage
\subsubsubsection{ Depth First Tree Tour (Construction)}
This algorithm (DFTT) based on minimum spanning tree (MST) or kruskal's algorithm, the length is exactly twice of the MST's weight, MST weight is not more than length of optimal tour, skipping visited nodes along the DFTT and apply Triangular Inequality, tour length at most twice of optimal length.\\
This flowchart explain all the process.\\
\begin{center}
	\includegraphics[width=12cm,height=13cm]{./assets/flowchart/dftt.png}\\
\end{center}
\newpage
Graph Example : \\
\begin{center}
	\includegraphics[width=9cm,height=8cm]{./assets/example/graph-dftt-1.png}\\
\end{center}
First get the MST of the graph like in the figure (in red).\\
\begin{center}
	\includegraphics[width=9cm,height=8cm]{./assets/example/graph-dftt-2.png}\\
\end{center}
Traverse the graph along MST edges and double the edges (in blue).\\
\begin{center}
	\includegraphics[width=9cm,height=8cm]{./assets/example/graph-dftt-3.png}\\
\end{center}
Last thing is to apply triangular inequality to get the best tour (in green).\\
\begin{center}
	\includegraphics[width=9cm,height=8cm]{./assets/example/graph-dftt-4.png}\\
\end{center}

\subsubsubsection{ 2-opt (Improvement)}
The 2-opt algorithm basically removes two cross over edges
from the tour, and reconnects the two paths created.
There is only one way to reconnect the two paths so that we still
have a valid tour. We do this only if the
new tour will be shorter. Continue removing and reconnecting
the tour until no 2-opt improvements can
be found. The tour is now 2-optimal. this figure explain cross over 2-opt can solve it.
\begin{center}
	\includegraphics[width=2cm,height=2cm]{./assets/example/graph-2opt-1.png}
\end{center}
This flowchart explain all the process.
\begin{center}
	\includegraphics[width=13cm,height=12cm]{./assets/flowchart/2-opt.png}\\
\end{center}

\subsubsection{ Genetic Algorithm Approach }
\begin{center}
	\includegraphics[width=17cm,height=13cm]{./assets/flowchart/ga.png}\\
\end{center}

\section{\sc System Analysis and Design}

\subsection{ System Architecture }
\begin{center}
	\includegraphics[width=16cm,height=13cm]{./assets/systemarchitecture/system-architecture.png}\\
\end{center}
\newpage

\subsection{ Stakeholders }

\subsubsection{  Customers }
Are External-Operational stakeholders, who make the orders and this stakeholders will interact with the system via Android application.

\subsubsection{  Provider }
Is Internal-Operational stakeholder, who receives the orders and distributes them to the distributors according to the address of the customers, this stakeholder will interact with the system via web application.

\subsubsection{  Distributers }
Is Internal-Operational stakeholder, receives the orders that they should deliver and ask the system to get the optimal route to follow, this stakeholders will interact with the system via Android application.
\newpage

\subsection{ Functional Requirements }

\subsubsection{ Customer }
\begin{itemize}
	\item Order any Product including quantity and location.
	\item Cancel order before distributor takes it.
	\item Update orders in cart.
\end{itemize}

\subsubsection{ Provider }
\begin{itemize}
	\item Reject or accept an order.
	\item Assign carts to distributor.
	\item Add distributor.
	\item Add admins.
\end{itemize}

\subsubsection{ Distributer }
\begin{itemize}
	\item View orders.
	\item Get/request the Order of delivering the orders based on the distance (The optimal path to go).
	\item View the path to the current fulfilled order.
	\item Confirm order delivery.
\end{itemize}
\newpage

\subsection{ Non-Functional Requirements }

\subsubsection{ Security }
\begin{itemize}
	\item The System has a form of protection by applying authorization, so any unauthorized access to the system is denied.
	\item Avoid SQL injection.
\end{itemize}

\subsubsection{ Performance }
\begin{itemize}
	\item Login must be completed in less than 3 seconds.
	\item Peak load 200 user every hour.
	\item Provider assign orders to distributor in less than 10 seconds.
\end{itemize}

\subsubsection{ Reliability }
\begin{itemize}
	\item The system has to be 100\% reliable.
\end{itemize}

\subsubsection{ Availability }
\begin{itemize}
	\item The system will be available 24/7.
\end{itemize}

\subsubsection{ Usability }
\begin{itemize}
	\item The customer can easily order any products with any quantities.
\end{itemize}

\subsection{ Use Case Diagram }
\begin{center}
	\includegraphics[width=17cm,height=13cm]{./assets/usecase/use-case-diagram.png}\\
\end{center}

\subsection{ Use Case Tables }

\subsubsection{ Customer }
\begin{center}
	\includegraphics[width=17cm,height=13cm]{./assets/usecasetable/customer-1.png}\\
	\includegraphics[width=17cm,height=13cm]{./assets/usecasetable/customer-2.png}\\
	\includegraphics[width=17cm,height=13cm]{./assets/usecasetable/customer-3.png}\\
\end{center}

\subsubsection{ Distributer }
\begin{center}
	\includegraphics[width=17cm,height=13cm]{./assets/usecasetable/distributer-1.png}\\
	\includegraphics[width=17cm,height=13cm]{./assets/usecasetable/distributer-2.png}\\
\end{center}

\subsubsection{ Provider }
\begin{center}
	\includegraphics[width=17cm,height=13cm]{./assets/usecasetable/provider-1.png}\\
\end{center}

\subsection{ Sequence Diagram }

\subsubsection{ Assign request to distributers }
\begin{center}
	\includegraphics[width=17cm,height=13cm]{./assets/sequencediagram/assignrequeststodistributers.png}\\
\end{center}

\subsubsection{ Order Work List }
\begin{center}
	\includegraphics[width=17cm,height=13cm]{./assets/sequencediagram/orderworklist.png}\\
\end{center}

\subsubsection{ Fulfill Order }
\begin{center}
	\includegraphics[width=17cm,height=13cm]{./assets/sequencediagram/fulfills.png}\\
\end{center}

\subsubsection{ Add products to cart }
\begin{center}
	\includegraphics[width=17cm,height=13cm]{./assets/sequencediagram/addproductstocart.png}\\
\end{center}

\subsubsection{ Open and edit cart }
\begin{center}
	\includegraphics[width=17cm,height=13cm]{./assets/sequencediagram/openandeditcart.png}\\
\end{center}
\newpage

\subsection{ Class Diagram }

\begin{center}
    \includegraphics[width=17cm,height=10.5cm]{./assets/class-diagram.png}
\end{center}
\newpage

\subsection{ ERD (Entity Relationship Diagram) }

\includegraphics[width=16.5cm,height=8.25cm]{./assets/erd.png}


\subsection{ Prototype }

\subsubsection{ Distributer }
\includegraphics[width=16cm,height=10cm]{./assets/prototype/distributer-1.png}\\

\subsubsection{ Provider }
\includegraphics[width=16.5cm,height=7cm]{./assets/prototype/provider-1.png}\\
\includegraphics[width=16.5cm,height=10cm]{./assets/prototype/provider-2.png}\\

\subsubsection{ Customer }
\includegraphics[width=16cm,height=10cm]{./assets/prototype/customer-1.png}\\

\subsection{ Gantt chart }

\includegraphics[width=17cm,height=8.5cm]{./assets/ganttchart/gantt-1.png}\\
\includegraphics[width=17cm,height=8.5cm]{./assets/ganttchart/gantt-2.png}\\
\includegraphics[width=17cm,height=8.5cm]{./assets/ganttchart/gantt-3.png}\\
\includegraphics[width=17cm,height=8.5cm]{./assets/ganttchart/gantt-4.png}\\
\includegraphics[width=17cm,height=8.5cm]{./assets/ganttchart/gantt-5.png}\\
\includegraphics[width=17cm,height=8.5cm]{./assets/ganttchart/gantt-6.png}\\

\section{\sc Conclusion}
We can see now that this project will help a lot by saving money, time and efforts for the companies that have a transportation system especially the ones with large transportation systems., as the application will choose the path with the most minimum distance approximately instead of travelling longer distances, and this is done by using one of the previously mentioned. 
In this project we considered one of the factors that affect the path to choose which is the distance between different requests which is constant factor, there are other factors that taking them into consideration will make better results like traveling time, traffic, the speed limit and the types of the streets chosen whether they are highways, freeways or small streets, taking all these factors well help in making the idea more effective, but it will make the project more complex ,so we can consider them later on as a better upgrade to the project for latter versions.

\end{document}